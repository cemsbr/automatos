\documentclass[brazil, a4paper,12pt]{article}
\usepackage[brazil]{babel}
\usepackage[utf8]{inputenc}
\usepackage[T1]{fontenc}
\usepackage{indentfirst}
\usepackage{geometry}
\usepackage{mathtools} % align
\usepackage{amssymb} % therefore
\geometry{a4paper,left=3cm,right=3cm,top=2.5cm,bottom=2.5cm}

% automata drawing
\usepackage{pgf}
\usepackage{tikz}
\usetikzlibrary{arrows,automata,positioning}

\begin{document}
\begin{titlepage}

  \vfill

  \begin{center}
    \begin{large}
      Universidade de São Paulo
    \end{large}
  \end{center}

  \begin{center}
    \begin{large}
      Instituto de Matemática e Estatística
    \end{large}
  \end{center}

  \begin{center}
    \begin{large}
      Programa de Pós-Graduação em Ciência da Computação
    \end{large}
  \end{center}

  \vfill

  \begin{center}
    \begin{Large}
        \textbf{MAC4722}\\
        \textbf{Linguagens, Autômatos e Computabilidade}\\
          Lista 4\\
    \end{Large}
  \end{center}

  \vfill

  \begin{center}
    \begin{large}
      Professor Arnaldo Mandel
    \end{large}
  \end{center}

  \begin{center}
    \begin{large}
      Carlos Eduardo Moreira dos Santos
    \end{large}
  \end{center}

  \vfill

  \begin{center}
    \begin{large}
      São Paulo \\
      \today \\
    \end{large}
  \end{center}

\clearpage
\end{titlepage}

\begin{enumerate}

%%%%%%%%%%%%%%%%%%%%%%%%%%%%%%%%%% Questão 1 %%%%%%%%%%%%%%%%%%%%%%%%%%%%%%%%%%
\item Construa gramáticas
  \begin{enumerate}
    \item $L = \{a^nb^mc^{2(m+n)} \mid m, n \geq 0\}$

    Seja a gramática $G$:

    \begin{enumerate}
      \item $S \rightarrow aScc \mid B$
      \item $B \rightarrow bBcc \mid \lambda$
    \end{enumerate}

    \item $L = \{w \in \{a,b\}^* \mid$ o número de $a$'s é o dobro do número de
    $b$'s$\}$

    Seja a gramática $G$:

    $S_0 \rightarrow S_1aab \mid aS_1ab \mid aaS_1b \mid aabS_1 \\
     S_0 \rightarrow S_1aba \mid aS_1ba \mid abS_1a \mid abaS_1 \\
     S_0 \rightarrow S_1baa \mid bS_1aa \mid baS_1a \mid baaS_1 \\
     S_1 \rightarrow S_0 \mid \lambda$

    A prova de que $L(G) \subseteq L$ segue por indução no comprimento da palavra.

    \begin{description}
      \item[Hipótese de indução:] $|w_n|_a = 2|w_n|_b$, onde $n$ é o comprimento de
    $w$ e $|w|_x$ é a quantidade de letras $x$ em $w$.

      \item[Base:] Para $n = 0$, temos $w = \lambda$ e $|w_0|_a = 0 = 2 \times 0 =
    2|w_0|_b$.

      \item[Passo indutivo:] O comprimento pode ser aumento por duas maneiras:
      \begin{itemize}
        \item Acrescentando 0 $a$'s e 0 $b$'s:
          \begin{align*}
            |w_n|_a &= |w_{n-1}|_a + 0 \\
            |w_n|_b &= |w_{n-1}|_b + 0 \\
            \\
            |w_{n-1}|_a &\overset{HI}{=} 2|w_{n-1}|_b \\
            |w_{n-1}|_a + 0 &= 2|w_{n-1}|_b + 0 \\
            |w_n|_a &= 2|w_n|_b
          \end{align*}

        \item Acrescentando 2 $a$'s e 1 $b$:
          \begin{align*}
            |w_n|_a &= |w_{n-1}|_a + 2 \\
            |w_n|_b &= |w_{n-1}|_b + 1 \\
            \\
            |w_{n-1}|_a &\overset{HI}{=} 2|w_{n-1}|\\
            |w_{n-1}|_a + 2 &= 2|w_{n-1}|_b + 2\\
            |w_{n-1}|_a + 2 &= 2(|w_{n-1}|_b + 1)\\
            |w_n|_a &= 2|w_n|_b
          \end{align*}
      \end{itemize}

    \end{description}

    Podemos notar que $L(G) \supseteq L$ pela construção da gramática. As três
    primeiras regras contêm todas as permutações de 2 $a$'s e 1 $b$, sendo cada
    uma em uma linha. É possível aumentar o comprimento da palavra em múltiplos
    de 3 através de $S_1$, colocado em todas as posições possíveis de cada
    permutação de $a$'s e $b$. Assim, todas as palavras de $L$ podem ser geradas
    pela gramática. Portanto, podemos concluir que $L(G) = L$.


  \item Disponível no arquivo em anexo: \textbf{string-repetida-complemento.pdf}
  \\ \\
  Resolução alternativa (André):\\
  \\
  $ S \rightarrow aSa \mid bSb \mid aSb \mid bSa \mid T \\
    T \rightarrow aRb \mid bRa \mid a \mid b \\
    R \rightarrow aRa \mid bRb \mid aRb \mid bRa \mid \lambda $ \\ \\
  A produção T garante que a palavra gerada tem $x_i \neq y_i$ ou que tem comprimento ímpar.


  \end{enumerate}

%%%%%%%%%%%%%%%%%%%%%%%%%%%%%%%%%% Questão 2 %%%%%%%%%%%%%%%%%%%%%%%%%%%%%%%%%%
\item Construa AP, de preferência determinístico
  \begin{enumerate}
    \item $\{0^n1^n2^k \mid n,k \geq 0\}$

    \begin{tikzpicture}[very thick, -stealth, shorten >=1pt,node distance=2cm,on
    grid,auto, scale=0.8]

    \node[state, initial, accepting] (0) at (0,0) {$q_0$};
    \node[state] (1) at (5, 0) {$q_1$};
    \node[state] (2) at (5, -5) {$q_2$};
    \node[state] (3) at (5, -10) {$q_3$};
    \node[state, accepting] (4) at (0, -5) {$q_4$};

    \path
    (0) edge node {$\lambda, \lambda \rightarrow \$$} (1)
    (1) edge node {$1, 0 \rightarrow \lambda$} (2)
        edge[loop right] node {$0, \lambda \rightarrow 0$} ()
    (2) edge node {$2, \lambda \rightarrow \lambda$} (3)
        edge[loop right] node {$1, 0 \rightarrow \lambda$} ()
        edge node[swap] {$\lambda, \$ \rightarrow \lambda$} (4)
    (3) edge node {$\lambda, \$ \rightarrow \lambda$} (4)
        edge[loop right] node {$2, \lambda \rightarrow \lambda$} ();
    \end{tikzpicture}
  \end{enumerate}

%%%%%%%%%%%%%%%%%%%%%%%%%%%%%%%%%% Questão 3 %%%%%%%%%%%%%%%%%%%%%%%%%%%%%%%%%%
\item

%%%%%%%%%%%%%%%%%%%%%%%%%%%%%%%%%% Questão 4 %%%%%%%%%%%%%%%%%%%%%%%%%%%%%%%%%%
\item

%%%%%%%%%%%%%%%%%%%%%%%%%%%%%%%%%% Questão 5 %%%%%%%%%%%%%%%%%%%%%%%%%%%%%%%%%%
\item

\end{enumerate}

\end{document}

% vim: set tw=80 fo=ct:
% set fo+=a for autowrap
