%%%%%%%%%%%%%%%%%%%%%%%%%%%%%%%%%%%%%%%%%
% Structured General Purpose Assignment
% LaTeX Template
%
% This template has been downloaded from:
% http://www.latextemplates.com
%
% Original author:
% Ted Pavlic (http://www.tedpavlic.com)
%
% Note:
% The \lipsum[#] commands throughout this template generate dummy text
% to fill the template out. These commands should all be removed when 
% writing assignment content.
%
%%%%%%%%%%%%%%%%%%%%%%%%%%%%%%%%%%%%%%%%%

%----------------------------------------------------------------------------------------
%	PACKAGES AND OTHER DOCUMENT CONFIGURATIONS
%----------------------------------------------------------------------------------------

\documentclass{article}

\usepackage{fancyhdr} % Required for custom headers
\usepackage{lastpage} % Required to determine the last page for the footer
\usepackage{extramarks} % Required for headers and footers
\usepackage{graphicx} % Required to insert images
\usepackage{lipsum} % Used for inserting dummy 'Lorem ipsum' text into the template

\usepackage[utf8]{inputenc}
\usepackage[brazilian]{babel}
\usepackage{amsmath}
\usepackage{amssymb}
\usepackage{enumerate}
\usepackage{array}

% For automata
\usepackage{pgf}
\usepackage{tikz}
\usetikzlibrary{arrows,automata}
\usetikzlibrary{shapes.multipart}

\newcommand{\qed}{\hfill \ensuremath{\Box}}

% Margins
\topmargin=-0.45in
\evensidemargin=0in
\oddsidemargin=0in
\textwidth=6.5in
\textheight=9.0in
\headsep=0.25in 

\linespread{1.1} % Line spacing

% Set up the header and footer
\pagestyle{fancy}
\lhead{\hmwkAuthorName} % Top left header
\chead{\hmwkClass\ (\hmwkClassInstructor): \hmwkTitle} % Top center header
\rhead{\firstxmark} % Top right header
\lfoot{\lastxmark} % Bottom left footer
\cfoot{} % Bottom center footer
\rfoot{Página\ \thepage\ de\ \pageref{LastPage}} % Bottom right footer
\renewcommand\headrulewidth{0.4pt} % Size of the header rule
\renewcommand\footrulewidth{0.4pt} % Size of the footer rule

\setlength\parindent{0pt} % Removes all indentation from paragraphs

%----------------------------------------------------------------------------------------
%	DOCUMENT STRUCTURE COMMANDS
%	Skip this unless you know what you're doing
%----------------------------------------------------------------------------------------

% Header and footer for when a page split occurs within a problem environment
\newcommand{\enterProblemHeader}[1]{
\nobreak\extramarks{#1}{#1 continua na próxima página\ldots}\nobreak
\nobreak\extramarks{#1 (continua)}{#1 continua na próxima página\ldots}\nobreak
}

% Header and footer for when a page split occurs between problem environments
\newcommand{\exitProblemHeader}[1]{
\nobreak\extramarks{#1 (continua)}{#1 continua na próxima página\ldots}\nobreak
\nobreak\extramarks{#1}{}\nobreak
}

\setcounter{secnumdepth}{0} % Removes default section numbers
\newcounter{homeworkProblemCounter} % Creates a counter to keep track of the number of problems

\newcommand{\homeworkProblemName}{}
\newenvironment{homeworkProblem}[1][Problem \arabic{homeworkProblemCounter}]{ % Makes a new environment called homeworkProblem which takes 1 argument (custom name) but the default is "Problem #"
\stepcounter{homeworkProblemCounter} % Increase counter for number of problems
\renewcommand{\homeworkProblemName}{#1} % Assign \homeworkProblemName the name of the problem
\section{\homeworkProblemName} % Make a section in the document with the custom problem count
\enterProblemHeader{\homeworkProblemName} % Header and footer within the environment
}{
\exitProblemHeader{\homeworkProblemName} % Header and footer after the environment
}

\newcommand{\problemAnswer}[1]{ % Defines the problem answer command with the content as the only argument
\noindent\framebox[\columnwidth][c]{\begin{minipage}{0.98\columnwidth}#1\end{minipage}} % Makes the box around the problem answer and puts the content inside
}

\newcommand{\homeworkSectionName}{}
\newenvironment{homeworkSection}[1]{ % New environment for sections within homework problems, takes 1 argument - the name of the section
\renewcommand{\homeworkSectionName}{#1} % Assign \homeworkSectionName to the name of the section from the environment argument
\subsection{\homeworkSectionName} % Make a subsection with the custom name of the subsection
\enterProblemHeader{\homeworkProblemName\ [\homeworkSectionName]} % Header and footer within the environment
}{
\enterProblemHeader{\homeworkProblemName} % Header and footer after the environment
}
   
%----------------------------------------------------------------------------------------
%	NAME AND CLASS SECTION
%----------------------------------------------------------------------------------------

\newcommand{\hmwkTitle}{3\textsuperscript{a} Lista de Exercícios} % Assignment title
\newcommand{\hmwkDueDate}{Quarta,\ 22\ de\ Maio,\ 2013} % Due date
\newcommand{\hmwkClass}{MAC\ 4722} % Course/class
\newcommand{\hmwkClassInstructor}{Arnaldo Mandel} % Teacher/lecturer
\newcommand{\hmwkAuthorName}{Andrew Kurauchi} % Your name

%----------------------------------------------------------------------------------------
%	TITLE PAGE
%----------------------------------------------------------------------------------------

\title{
\textmd{\textbf{\hmwkClass:\ \hmwkTitle}}\\
}

\author{\textbf{\hmwkAuthorName} - nusp: 5894035}
\date{} % Insert date here if you want it to appear below your name

%----------------------------------------------------------------------------------------

\begin{document}

\maketitle

%----------------------------------------------------------------------------------------

\section{Exercício 1}

Construa gramáticas para as linguagens:

\begin{enumerate}[(a)]
	\item $\{a^nb^mc^{2(n+m)}|m,n\geq 0\}$
	\item $\{w\in\{a,b\}^*|\text{o número de }a\text{'s é o dobro do número de }b\text{'s}\}$
	\item $\{w\in\{a,b\}^*|w\text{ não é da forma }xx\}$
\end{enumerate}

\subsection{Resolução}

\subsubsection{(a)}

Considere a gramática $G = (V, \Sigma, P, S)$ onde:

\begin{itemize}
	\item $V = \{S, B, a, b, c\}$
	\item $\Sigma = \{a, b, c\}$
	\item $P:$
	\begin{itemize}
		\item $S\rightarrow aScc | B$
		\item $B\rightarrow bBcc | \lambda$
	\end{itemize}
\end{itemize}

% TODO falta mostrar que reconhece a linguagem

\subsubsection{(b)}

Considere a gramática $G = (V, \Sigma, P, S)$ onde:

\begin{itemize}
	\item $V = \{S, a, b\}$
	\item $\Sigma = \{a, b\}$
	\item $P=\{S\rightarrow SaSaSbS | SaSbSaS | SbSaSaS | \lambda\}$
\end{itemize}

% TODO falta mostrar que reconhece a linguagem

\subsubsection{(c)}

% TODO

\section{Exercício 2}

Para as linguagens abaixo, construa autômato com pilha determinístico, ou um não determinístico, mas nesse caso explique por que não conseguiu fazer determinístico:

\begin{enumerate}[(a)]
	\item $\{0^n1^n2^k|n,k\geq 0\}$
	\item $\{w\in\{a,b\}^*|w\text{ tem um prefixo com mais }a\text{'s do que }b\text{'s}\}$
	\item $\{w\in\{a,b\}^*|w\text{ tem um sufixo com mais }a\text{'s do que }b\text{'s}\}$
\end{enumerate}

\subsection{Resolução}

\subsubsection{(a)}

\begin{center}
\begin{tikzpicture}[->,>=stealth',shorten >=1pt,auto,node distance=2.8cm,
                    semithick]

	\node[initial,state]     (A)              {};
	\node[state]             (B) [right of=A] {};
	\node[state,accepting]   (C) [right of=B] {};


	\path (A) edge [loop above]  node {0,$\lambda$/0} (A)
			  edge []			 node {1,0/$\lambda$} (B)
		  (B) edge [loop above]  node {1,0/$\lambda$} (B)
			  edge []		     node {2,$\lambda/\lambda$} (C)
		  (C) edge [loop above]  node {2,$\lambda/\lambda$} (C);
\end{tikzpicture}
\end{center}

% TODO mostrar que reconhece L

\subsubsection{(b)}

\begin{center}
\begin{tikzpicture}[->,>=stealth',shorten >=1pt,auto,node distance=2.8cm,
                    semithick]

	\node[initial,state]     (A)              {};
	\node[state]             (B) [right of=A] {};
	\node[state,accepting]   (C) [right of=B] {};


	\path (A) edge []			 node {$\lambda$,$\lambda/0$} (B)
		  (B) edge [loop above]  node {$a,b/\lambda$} (B)
			  edge [loop below]  node {$b,\lambda/b$} (B)
			  edge []		     node {$a,0/\lambda$} (C)
		  (C) edge [loop above]  node {$a,\lambda/\lambda$} (C)
			  edge [loop below]  node {$b,\lambda/\lambda$} (C);
\end{tikzpicture}
\end{center}

% TODO mostrar que reconhece L

\subsubsection{(c)}

\begin{center}
\begin{tikzpicture}[->,>=stealth',shorten >=1pt,auto,node distance=2.8cm,
                    semithick]

	\node[initial,state]     (A)              {};
	\node[state,accepting]   (B) [right of=A] {};


	\path (A) edge [loop above]	 node {$b,\lambda/\lambda$} (A)
			  edge [bend left]   node {$a,\lambda/0$} (B)
		  (B) edge [loop above]  node {$a,\lambda/a$} (B)
			  edge [loop below]  node {$b,a/\lambda$} (B)
			  edge [bend left]   node {$b,0/\lambda$} (A);
\end{tikzpicture}
\end{center}

% TODO mostrar que reconhece L

\section{Exercício 3}

Construa MT para as seguintes linguagens (explicando como funcionam):

\begin{enumerate}[(a)]
	\item $\{a^mb^n|m\neq n\}$
	\item $\{a^nb^nc^n|n\geq 0\}$
	\item $\{ww|w\in\{a,b\}^*\}$
\end{enumerate}

\subsection{Resolução}

\subsubsection{(a)}

\begin{center}
\begin{tikzpicture}[->,>=stealth',shorten >=1pt,auto,node distance=2.8cm,
                    semithick]

	\node[initial,initial where=below,state]     (A)              {};
	\node[state]		     (B) [right of=A, xshift=1cm] {};
	\node[state]			 (E) [above right of=B] {};
	\node[state]			 (C) [below right of=E] {};
	\node[state]			 (D) [left of=A] {};
	\node[state]			 (G) [left of=D] {};
	\node[state,accepting]   (F) [above of=D] {$q_\text{aceita}$};


	\path (A) edge				node {$a\rightarrow a', D$} (B)
			  edge [bend right] node [swap]{$b\rightarrow b, D$} (G)
			  edge				node {$b'\rightarrow b', D$} (D)
		  (B) edge [loop above] node {$a\rightarrow a, D$} (B)
			  edge				node [swap,pos=0.6]{$b'\rightarrow b', D$} (E)
			  edge				node {$b\rightarrow b', E$} (C)
			  edge				node [swap]{\textvisiblespace$\rightarrow$\textvisiblespace, $D$} (F)
		  (C) edge [loop above] node {$\begin{matrix}a\rightarrow a, E\\b'\rightarrow b', E\end{matrix}$} (C)
			  edge [bend left]  node {$a'\rightarrow a', D$} (A)
		  (D) edge [loop below] node {$b'\rightarrow b', D$} (G)
			  edge				node {$b\rightarrow b, D$} (G)
		  (E) edge [loop above] node {$b'\rightarrow b', D$} (E)
			  edge				node [swap]{\textvisiblespace$\rightarrow$\textvisiblespace, $D$} (F)
			  edge				node [pos=0]{$b\rightarrow b', E$} (C)
		  (G) edge [loop below] node {$b\rightarrow b, D$} (G)
			  edge				node {\textvisiblespace$\rightarrow$\textvisiblespace, $D$} (F);
\end{tikzpicture}
\end{center}

% TODO mostrar que reconhece L

\subsubsection{(b)}

\begin{center}
\begin{tikzpicture}[->,>=stealth',shorten >=1pt,auto,node distance=2.8cm,
                    semithick]

	\node[initial,state]     (A)              {};
	\node[state]		     (B) [right of=A] {};
	\node[state]		     (C) [right of=B] {};
	\node[state]		     (D) [right of=C] {};
	\node[state,accepting]   (E) [above of=A] {$q_\text{aceita}$};


	\path (A) edge [loop below]	 node {$\begin{matrix}a'\rightarrow a',D\\b'\rightarrow b',D\\c'\rightarrow c',D\end{matrix}$} (A)
			  edge				 node {$a\rightarrow a', D$} (B)
			  edge				 node {$\text{\textvisiblespace}\rightarrow \text{\textvisiblespace}, E$} (E)
		  (B) edge [loop above]  node {$\begin{matrix}a\rightarrow a,D\\b'\rightarrow b',D\end{matrix}$} (B)
			  edge				 node {$b\rightarrow b',D$} (C)
		  (C) edge [loop above]  node {$\begin{matrix}b\rightarrow b,D\\c'\rightarrow c',D\end{matrix}$} (C)
			  edge				 node {$c\rightarrow c',E$} (D)
		  (D) edge [loop above]  node {$\begin{matrix}a\rightarrow a,E\\b\rightarrow b,E\\b'\rightarrow b',E\\c'\rightarrow c',E\end{matrix}$} (D)
			  edge [bend left]   node {$a'\rightarrow a',D$} (A);
\end{tikzpicture}
\end{center}

% TODO mostrar que reconhece L
% Em D não ocorre c, pois já foi pra esquerda
% Em A tudo a esquerda já foi contado

\subsubsection{(c)}

% Não vou resolver porque vai ficar muito grande
% 1) Primeiro encontra o meio - marca o primeiro, vai até o último e marca. Volta até o último marcado, marca e vai até o último não marcado, etc.
% 2) Se não encontra o par então a palavra não é ww e pode rejeitar a palavra, senão marca o meio e volta para o começo
% 3) Marca o primeiro e vai até o último marcado. Marca o próximo e volta. Se for igual ao primeiro volta, senão rejeita.
% 4) Se chegar no último (espaço), verifica se a última letra é igual e decide.

\section{Exercício 4}

Construa MT que:

\begin{enumerate}[(a)]
	\item Copia o conteúdo da fita 1 na fita 2
	\item Procure na fita 1 o texto $(w,$, onde $w$ é o conteúdo da fita 2, e os símbolos $(,$ não ocorrem em $w$.
\end{enumerate}

\subsection{Resolução}

Utilizaremos a seguinte notação para indicar uma transição: $\sigma_1,sigma_2\rightarrow (\sigma_3, d_1),(\sigma_4, d_2)$ onde $\sigma_1$ é o caracter na cabeça de leitura na fita 1, e $\sigma_2$ é o caracter na cabeça de leitura na fita 2, que são substituídos por $\sigma_3$ e $\sigma_4$ respectivamente e a cabeça de leitura da fita 1 se move para $d_1$ e a cabeça da fita 2 se move para $d_2$, $d_1, d_2\in\{E,D,F\}$ ($F$ indica que a cabeça permanece no mesmo local).

\subsubsection{(a)}

\begin{center}
\begin{tikzpicture}[->,>=stealth',shorten >=1pt,auto,node distance=4.8cm,
                    semithick]

	\node[initial,state]   (A)              {};
	\node[state,accepting] (B) [right of=A] {$q_\text{aceita}$};

	\path (A) edge [loop above]	 node {$\sigma,\text{\textvisiblespace}\rightarrow(\sigma,D),(\sigma,D), \sigma\in\Gamma$} (A)
			  edge				 node {$\text{\textvisiblespace,\textvisiblespace}\rightarrow (\text{\textvisiblespace},D),(\text{\textvisiblespace},D)$} (B);
\end{tikzpicture}
\end{center}

% TODO Mostrar que faz o que promete

\subsubsection{(b)}

\begin{center}
\begin{tikzpicture}[->,>=stealth',shorten >=1pt,auto,node distance=5.3cm,
                    semithick]

	\node[initial,state]   (A)              {};
	\node[state]		   (B) [right of=A] {};
	\node[state]		   (C) [below of=B,yshift=3.5cm] {};
	\node[state,accepting] (D) [right of=B] {$q_\text{aceita}$};

	\path (A) edge [loop above]	 node {$\sigma_1,\sigma_2\rightarrow(\sigma_1,D),(\sigma_2,F), \sigma_1\neq (, \sigma_2\in\Sigma$} (A)
			  edge				 node {$(,\sigma\rightarrow ((,D),(\sigma,F), \sigma\in\Sigma$} (B)
		  (B) edge [loop above]	 node {$\sigma,\sigma\rightarrow(\sigma,D),(\sigma',D), \sigma\in\Sigma$} (B)
			  edge				 node {$,,\text{\textvisiblespace}\rightarrow (,,D),(\text{\textvisiblespace},D)$} (D)
			  edge				 node {$\sigma_1,\sigma_2\rightarrow (\sigma_1,F),(\sigma_2,E),\sigma_1\neq\sigma_2,\sigma_1\in\Gamma,\sigma_2\in\Sigma$} (C)
		  (C) edge [loop below]  node {$\sigma_1,\sigma_2'\rightarrow (\sigma_1,F),(\sigma_2,E),\sigma_1\in\Gamma,\sigma_2\in\Sigma$} (C)
			  edge				 node {$\sigma_1,\sigma_2\rightarrow (\sigma_1,F),(\sigma_2,F),\sigma_1\in\Gamma,\sigma_2\in\Sigma$} (A);
\end{tikzpicture}
\end{center}

Onde $\sigma'$ é a letra $\sigma$ com uma marcação, $\Sigma$ é o alfabeto de entrada, $\Gamma=\Sigma\cup\{\sigma'|\sigma\in\Sigma\}\cup\{(,,,\text{\textvisiblespace}\}, \Sigma\cap\{\sigma'|\sigma\in\Sigma\} = \emptyset$.

% TODO Mostrar que faz o que promete

\section{Exercício 5}
Considere MT que, a cada configuração, ou podem escrever um símbolo ou podem mover a cabeça de leitura.

\begin{enumerate}[(a)]
	\item Escreva uma definição formal desse tipo de MT
	\item Mostre que as linguagens reconhecidas por essas MT são exatamente as linguagens Turing-reconhecíveis.
\end{enumerate}

\subsection{Resolução}

\subsubsection{(a)}

Uma MT com essas restrições é definida pela quíntupla $M=(Q,s,\Sigma,\Gamma,\delta)$:

\begin{itemize}
	\item $Q$: conjunto finito de estados;
	\item $s\in Q$: estado inicial;
	\item $\Sigma$: alfabeto de entrada;
	\item $\Gamma$: alfabeto da fita: $\Sigma\subseteq\Gamma,\text{\textvisiblespace}\in\Gamma\setminus\Sigma$;
	\item $\delta: Q\times\Gamma\rightarrow Q\times\Gamma\times\{E,D,F\}$ tal que $\delta(p,\sigma) = (q,\tau,F), \sigma,\tau\in\Gamma$ ou $\delta(p,\sigma) = (q,\sigma,d), d\in\{E,D\}$.
\end{itemize}

\subsubsection{(b)}

Como essas MTs são um subconjunto de MT então basta mostrar que qualquer MT pode ser descrita com essas restrições. Considere uma transição qualquer: $\delta(p,\sigma) = (q,\tau,d)$. O mesmo efeito é obtido utilizando as restrições da seguinte forma (só é necessário considerar o caso em que $d\in\{E,D\}$): 

\begin{itemize}
	\item Adicionamos um estado $r$, $r\not\in Q$, intermediário que será utilizado somente para essa transição;
	\item $\delta(p,\sigma) = (r,\tau,F)$;
	\item $\delta(r,\tau) = (q,\tau,d)$.
\end{itemize}

\section{Exercício 6}

Construa MT's que computem as seguintes funções (convenção: a entrada deve estar na fita 1 - pode ter qualquer número de fitas - e a saída é o conteúdo da fita 1 na parada):

\begin{enumerate}[(a)]
	\item $f(w) = ww$
	\item $f(n) = 2^n$ - isto são dois problemas:
	\begin{enumerate}[i.]
		\item Números expressos em unário: $n$ é dado como $a^n$
		\item Números expressos em binário
	\end{enumerate}
\end{enumerate}

\subsection{Resolução}

\subsubsection{(a)}

\begin{center}
\begin{tikzpicture}[->,>=stealth',shorten >=1pt,auto,node distance=4.8cm,
                    semithick]

	\node[initial,state]   (A)              {};
	\node[state]		   (B) [right of=A] {};
	\node[state]		   (C) [below of=A, yshift=3cm] {};
	\node[state]		   (D) [right of=C] {};
	\node[state,accepting] (E) [right of=D] {$q_\text{aceita}$};

	\path (A) edge [loop above]	 node {$\sigma,\text{\textvisiblespace}\rightarrow(\sigma',D),(\sigma,D), \sigma\in\Gamma$} (A)
			  edge				 node {$\text{\textvisiblespace,\textvisiblespace}\rightarrow (\text{\textvisiblespace},E),(\text{\textvisiblespace},E)$} (B)
		  (B) edge [loop above]  node {$\sigma',\sigma\rightarrow (\sigma,E),(\sigma,E), \sigma\in\Sigma$} (B)
			  edge				 node[pos=0.25] {$\sigma,\sigma\rightarrow (\sigma,F),(\sigma,F), \sigma\in\Sigma$} (C)
		  (C) edge [loop below]  node {$\sigma_1,\sigma_2\rightarrow (\sigma_1,D),(\sigma_2,F)$} (C)
			  edge				 node {\textvisiblespace,$\sigma\rightarrow (\sigma,D),(\sigma,D)$} (D)
		  (D) edge [loop below]  node {\textvisiblespace,$\sigma\rightarrow (\sigma,D),(\sigma,D)$} (D)
			  edge				 node {\textvisiblespace,\textvisiblespace$\rightarrow$ (\textvisiblespace,$F$),(\textvisiblespace,$F$)} (E);
\end{tikzpicture}
\end{center}

Onde $\sigma'$ é a letra $\sigma$ com uma marcação, $\sigma'\not\in\Sigma$.

% TODO Mostrar que faz o que promete

\subsubsection{(b.i.)}

Utilizaremos a MT construída no item anterior representada por um quadrado. Transições que chegam no quadrado vão para o estado inicial e transições que saem do quadrado saem do estado de aceitação (que não é mais final). Além disso a segunda fita utilizada pela MT auxiliar é diferente da segunda fita utilizada por esta MT (por simplicidade não mostraremos o que acontece na fita que não está sendo utilizada, pois sempre fica parada). Supomos que na MT auxiliar sempre ocorre um pré-processamento que leva a cabeça de leitura das duas fitas de volta para a posição inicial.

\begin{center}
\begin{tikzpicture}[->,>=stealth',shorten >=1pt,auto,node distance=4.8cm,
                    semithick]

	\node[initial,state]			(A)              {};
	\node[state]					(B) [right of=A] {};
	\node[state]					(C) [below of=A, yshift=2cm] {};
	\node[state,shape=rectangle]	(D) [below of=C, yshift=2cm] {};
	\node[state,accepting]			(E) [right of=C] {$q_\text{aceita}$};

	\path (A) edge [loop above]	 node {$a,\text{\textvisiblespace}\rightarrow(a,D),(a,D)$} (A)
			  edge				 node {$\text{\textvisiblespace,\textvisiblespace}\rightarrow (\text{\textvisiblespace},E),(\text{\textvisiblespace},E)$} (B)
		  (B) edge [loop above]  node {$a,a\rightarrow ($\textvisiblespace$,E),(a,F)$} (B)
			  edge				 node[pos=0.25] {$\begin{matrix}\text{\textvisiblespace},a\rightarrow (a,D),(a,F)\\\text{\textvisiblespace,\textvisiblespace}\rightarrow (a,D),(\text{\textvisiblespace},F)\end{matrix}$} (C)
		  (C) edge [bend right]  node [swap]{\textvisiblespace,$a\rightarrow ($\textvisiblespace$,E),(a,F)$} (D)
			  edge				 node {\textvisiblespace,\textvisiblespace$\rightarrow ($\textvisiblespace$,F),($\textvisiblespace$,F)$} (E)
		  (D) edge [bend right]  node [swap]{\textvisiblespace,$a\rightarrow ($\textvisiblespace$,F),($\textvisiblespace$,E)$} (C);
\end{tikzpicture}
\end{center}

% TODO Mostrar que faz o que promete

\subsubsection{(b.ii.)}

% Não vou desenhar a MT porque vai ficar muito grande

% Utilizaremos 4 fitas
% Fita 1: Principal
% Fita 2: Cópia da principal
% Fita 3: Índica da potência de 2 atual em unário
% Fita 4: Cópia da fita 3

% Copia da fita 1 para a fita 2 apagando a fita 1. Coloca 1 na fita 1. Vai até a direita da fita 2. Para cada letra (0,1) coloca um 0 na fita 3 e a apaga da fita 2 indo para a esquerda. Se for 1 copia o conteúdo da fita 3 para a 4 e a usa como fita principal da MT do item (a). Depois copia o resultado na fita 4 para a direita da fita 1. Quando a fita 2 está vazia acaba o programa.

\end{document}
