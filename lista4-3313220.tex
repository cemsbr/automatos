\documentclass[brazil, a4paper,12pt]{article}
\usepackage[brazil]{babel}
\usepackage[utf8]{inputenc}
\usepackage[T1]{fontenc}
\usepackage{indentfirst}
\usepackage{geometry}
\usepackage{amssymb} % therefore
\geometry{a4paper,left=3cm,right=3cm,top=2.5cm,bottom=2.5cm}

\begin{document}
\begin{titlepage}

  \vfill

  \begin{center}
    \begin{large}
      Universidade de São Paulo
    \end{large}
  \end{center}

  \begin{center}
    \begin{large}
      Instituto de Matemática e Estatística
    \end{large}
  \end{center}

  \begin{center}
    \begin{large}
      Programa de Pós-Graduação em Ciência da Computação
    \end{large}
  \end{center}

  \vfill

  \begin{center}
    \begin{Large}
        \textbf{MAC4722}\\
        \textbf{Linguagens, Autômatos e Computabilidade}\\
          Lista 4\\
    \end{Large}
  \end{center}

  \vfill

  \begin{center}
    \begin{large}
      Professor Arnaldo Mandel
    \end{large}
  \end{center}

  \begin{center}
    \begin{large}
      Carlos Eduardo Moreira dos Santos
    \end{large}
  \end{center}

  \vfill

  \begin{center}
    \begin{large}
      São Paulo \\
      \today \\
    \end{large}
  \end{center}

\clearpage
\end{titlepage}

\begin{enumerate}

%%%%%%%%%%%%%%%%%%%%%%%%%%%%%%%%%% Questão 1 %%%%%%%%%%%%%%%%%%%%%%%%%%%%%%%%%%
\item Construa gramáticas
  \begin{enumerate}
    \item $L = \{a^nb^mc^{2(m+n)} \mid m, n > 0\}$

    Seja a gramática $G$:

    \begin{enumerate}
      \item $S \rightarrow aScc$ \label{regra1a1}
      \item $S \rightarrow aBcc$ \label{regra1a2}
      \item $B \rightarrow bBcc$ \label{regra1a3}
      \item $B \rightarrow bcc$ \label{regra1a4}
    \end{enumerate}

    A menor derivação possível é $S \Rightarrow aBcc \Rightarrow abcccc$ que
    gera a menor palavra de $L$ ($m, n = 1 > 0$). A última regra produz
    $bcc$, utilizado na terceira para gerar palavras do tipo $b^mc^{2m}$. A
    primeira regra torna possível prefixar $a^n$ e sufixar $c^{2n}$. Por
    fim, a segunda regra permite inserir $b^mc^{2m}$ entre um $a$ e um $c$
    de $a^nb^{2n}$. Dessa forma, obtemos $a^nb^mc^{2m}c^{2n} =
    a^nb^mc^{2(m+n)} \in L$.

  \end{enumerate}

%%%%%%%%%%%%%%%%%%%%%%%%%%%%%%%%%% Questão 2 %%%%%%%%%%%%%%%%%%%%%%%%%%%%%%%%%%
\item

%%%%%%%%%%%%%%%%%%%%%%%%%%%%%%%%%% Questão 3 %%%%%%%%%%%%%%%%%%%%%%%%%%%%%%%%%%
\item

%%%%%%%%%%%%%%%%%%%%%%%%%%%%%%%%%% Questão 4 %%%%%%%%%%%%%%%%%%%%%%%%%%%%%%%%%%
\item

%%%%%%%%%%%%%%%%%%%%%%%%%%%%%%%%%% Questão 5 %%%%%%%%%%%%%%%%%%%%%%%%%%%%%%%%%%
\item

\end{enumerate}

\end{document}

% vim: set tw=80 fo=ct:
% set fo+=a for autowrap
